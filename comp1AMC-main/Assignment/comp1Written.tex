\documentclass{article}
\usepackage{graphicx} % Required for inserting images

\title{Composition 1: Blues Harmony Generator}
\author{Jake Kerrigan, Milo Goldstein}

\date{February 2025 - 10 Hours}

\begin{document}

\maketitle

\section{How To Run}

Comp1.py can be ran in three modes. 
\begin{enumerate}
  \item No flag is given (i.e just python3 comp1.py) then the program will output a text-based representation of the generated 12-bar blues.
  \item If the -m flag is given then the program will attempt to play the 12-bar blues using the system's default midi player.
  \item If the -s flag is given then the program will attempt to graphically display the sheet music.
\end{enumerate}

Once you start the program you will be prompted to enter three settings: key, number of rounds of derivations, and min chord length.
After this, the program will generate the 12-bar blues. Then, based on the flag provided to the command line, it will output you the corresponding format of the 12 bar blues. 

\section{Description of Composition}

Generating our composition took a combination of randomization and both grammar-based and serialization-based approaches:\\
\\
The harmony used a modified implementation of Steedman's grammar in his paper \textit{A Generative Grammar for Jazz Chord Sequences}, which we learned about in class. We modified this grammar's implementation based on our previous knowledge of jazz blues progressions and trends within our favorite jazz blues pieces. The last (two to) four bars of a jazz blues are known as the piece's turnaround. This is often a sequence of, potentially tri-tone substituted, 2-5-1's. In fact, much of that "jazz sound" can be achieved by a combination of 2-5-1's, or their tri-tone substitutions, regardles of where they exist in the piece. Thus, we kept all rules directly implimenting 2-5's or their substitutions in our code: rules 3 and 4, with 3 modified to be one single rule where the quality of the dominant is dependent on the quality of x. We also kept rules 1 and 2 since they demonstrate pretty classic blues tropes. Given that faster changing harmonic rhythm tends towards the end of jazz blues pieces (in both that turnaround section, and the bar before it which may itsself contain a fast moving 2-5 transition into the turnaround), we used an end-of-piece-skewed randomization approach to decide where to apply derivations.\\
\\
The melody used a modified serialization-based approach to generate rhythms and a randomized approach to generating pitches for each rhythmic note. The serialization worked only on 5 different outputs (quarter note, quarter rest, two eighth notes, eighth note \& eighth rest, eighth rest \& eighth note). The prime, instead of containing one of each rhythm, contained 16 rhythms randomly selected from our list of 5. From here, 5 transformations are applied on the Prime. Each transformation is a random choice from one of the four class-learned transformations. From there, pitches in the user-inputted key's corresponding minor blues scale are applied to the previously-generated rhythms.

\section{Visualization/Demonstration}

\begin {figure}[h]
\rotatebox[origin=c]{0}{\includegraphics[width=13cm]{grammer.png}}
\centering
\end{figure}

Our composition uses the above grammar to derive on the standard 12 bar blues harmony (as defined in rule 0). (Note we decided to combine rules 3a and 3b into one rule and made the first chord of the RHS minor 50 percent of the time) The below output represents 1 "round" of derivation in the key of C, where each round represents applying each rule once to the current 12 bar blues phrase.


\begin {figure}[h]
\rotatebox[origin=c]{0}{\includegraphics[width=13cm]{simpleOutput.pdf}}
\centering
\end{figure}

Unfortunately, some of the derivations can cancel each other out. Derivation 3 and 4 both effectively delete a chord, w, so seeing where each derivation was applied can be difficult. One clear example in the below output is measure 8. This is an application of rule 4; we can see that the program replaced the chord in measure 8 with the dominate flat supertonic relative to G dominant (in other words, the tri-tone sub of the 5th of G). Rule 4 was applied here because the following chord is a 7th chord and before application the replaced chord must have been a dominate 5th of the G (D dominate) which was likely generated by rule 3 (which generates 5-1s)!
    
\section{Reflection}

Many blues songs contain melodies of the form (4 bars melody1) (4 bars melody1) (4 bars melody2). Since this style both emphasizes the turnaround tonally for the listener, and it is a pleasing sound to us, we implemented our melody in this fashion. In the future, having rules about what notes 'should come next' based on past notes (i.e. following the blue note with the fourth or sixth note of the minor blues scale) may lend itself to more consistent 'nice' resolutions in the melody than a randomized approach. Furthermore, using a repertoire of chord-based scales to decide what the next melodic pitch should be would provide a wider range of 'nice' sounds for our piece. This concept can be taken to the nth degree where melodic pitches could be informed by chord tones, possible scales, color notes, and the note-context surrounding them. If we had infinite time, we'd do something like this, which is akin to impro-Visor. 

An improvement to the harmonization would be to change the voicings/inversions of chords. Currently all chords are in root positions, but by changing that we could make the harmony sound less angular and jumpy by making the space between each chord smaller.

\section{What Gave Us Trouble?}

In general combining streams in music21 can be troublesome, overall our workaround just involved flattening the stream before outputting it.
However, what gave us the most trouble in this assignment was trying to implement applyDerive3. Specifically, the functionality to cut out two chords from a larger stream and then insert in two new ones proved to have multiple ways to mess up the overall stream. Especially because seeing where the generation went wrong was very hard in testing with multiple rounds of derivations. But after many sessions and rewrites we were able to get the function to behave correctly on every generation. (But let us know if you come across any malformatted output)!

The roman.Roman chord subclass gave us some trouble as well. The notation for defining chords in the roman subclass is a bit more extended than that of the chord subclass. In roman, symbols such as "b" and the capitalization of the numeral(s) have an impact on the chord. Also, symbols such as IV7 applied in different keys do not hold the same context. Say we apply this in c major, we'll get an F major7 chord (since, diatonically, the e is natural in c major). If we used iv7, though, we'd get F minor7 (with the b flat). If we wanted that b flat in the original IV7, we could have done a few things (and there may even be more ways to do this): IV7[b7] or IV-7. Both of these variations give us the flat 7, as opposed to the major 7.

\end{document}
